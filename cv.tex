
\documentclass[10pt]{article} % Default font size

%%%%%%%%%%%%%%%%%%%%%%%%%%%%%%%%%%%%%%%%%
% Glick CV
% Structure Specification File
% Version 1.0 (11/10/2015)
%
%%%%%%%%%%%%%%%%%%%%%%%%%%%%%%%%%%%%%%%%%

%----------------------------------------------------------------------------------------
%	PACKAGES AND OTHER DOCUMENT CONFIGURATIONS
%----------------------------------------------------------------------------------------

\usepackage[a4paper, hmargin=25mm, vmargin=30mm, top=20mm]{geometry} % Use A4 paper and set margins

\usepackage{fancyhdr} % Customize the header and footer

\usepackage{lastpage} % Required for calculating the number of pages in the document

\usepackage{hyperref} % Colors for links, text and headings

\setcounter{secnumdepth}{0} % Suppress section numbering

%\usepackage[proportional,scaled=1.064]{erewhon} % Use the Erewhon font
%\usepackage[erewhon,vvarbb,bigdelims]{newtxmath} % Use the Erewhon font
\usepackage[utf8]{inputenc} % Required for inputting international characters
\usepackage[T1]{fontenc} % Output font encoding for international characters

\usepackage[proportional,scaled=1.064]{erewhon}
\usepackage[erewhon,vvarbb,bigdelims]{newtxmath}
\usepackage[T1]{fontenc}
\renewcommand*\oldstylenums[1]{\textosf{#1}}

\usepackage{color} % Required for custom colors
\definecolor{slateblue}{rgb}{0.17,0.22,0.34}

\usepackage{sectsty} % Allows customization of titles
\sectionfont{\color{slateblue}} % Color section titles

\fancypagestyle{plain}{\fancyhf{}\cfoot{\thepage\ of \pageref{LastPage}}} % Define a custom page style
\pagestyle{plain} % Use the custom page style through the document
\renewcommand{\headrulewidth}{0pt} % Disable the default header rule
\renewcommand{\footrulewidth}{0pt} % Disable the default footer rule

\setlength\parindent{0pt} % Stop paragraph indentation

% Non-indenting itemize
\newenvironment{itemize-noindent}
{\setlength{\leftmargini}{0em}\begin{itemize}}
{\end{itemize}}

% Text width for tabbing environments
\newlength{\smallertextwidth}
\setlength{\smallertextwidth}{\textwidth}
\addtolength{\smallertextwidth}{-2cm}

\newcommand{\sqbullet}{~\vrule height 1ex width .8ex depth -.2ex} % Custom square bullet point definition

%----------------------------------------------------------------------------------------
%	MAIN HEADER COMMAND
%----------------------------------------------------------------------------------------

\renewcommand{\title}[1]{
{\huge{\color{slateblue}\textbf{#1}}}\\ % Header section name and color
\rule{\textwidth}{0.5mm}\\ % Rule under the header
}

%----------------------------------------------------------------------------------------
%	JOB COMMAND
%----------------------------------------------------------------------------------------

\newcommand{\job}[6]{
\begin{tabbing}
\hspace{2cm} \= \kill
\textbf{#1} \> \href{#4}{#3} \\
\textbf{#2} \>\+ \textit{#5} \\
\begin{minipage}{\smallertextwidth}
\vspace{2mm}
#6
\end{minipage}
\end{tabbing}
\vspace{2mm}
}

\newcommand{\award}[6]{
	\begin{tabbing}
		\hspace{2cm} \= \kill
		\textbf{#1} \> \href{#4}{#3} \\
		\textbf{#2} \>\+ \textit{#5} \\
		\begin{minipage}{\smallertextwidth}
			\vspace{2mm}
			#6
		\end{minipage}
	\end{tabbing}
	\vspace{2mm}
}
%----------------------------------------------------------------------------------------
%	SKILL GROUP COMMAND
%----------------------------------------------------------------------------------------

\newcommand{\skillgroup}[2]{
\begin{tabbing}
\hspace{5mm} \= \kill
\sqbullet \>\+ \textbf{#1} \\
\begin{minipage}{\smallertextwidth}
\vspace{2mm}
#2
\end{minipage}
\end{tabbing}
}

%----------------------------------------------------------------------------------------
%	INTERESTS GROUP COMMAND
%-----------------------------------------------------------------------------------------

\newcommand{\interestsgroup}[1]{
\begin{tabbing}
\hspace{5mm} \= \kill
#1
\end{tabbing}
\vspace{-10mm}
}

\newcommand{\interest}[1]{\sqbullet \> \textbf{#1}\\[3pt]} % Define a custom command for individual interests

%----------------------------------------------------------------------------------------
%	TABBED BLOCK COMMAND
%----------------------------------------------------------------------------------------

\newcommand{\tabbedblock}[1]{
\begin{tabbing}
\hspace{2.3cm} \= \hspace{4cm} \= \kill
#1
\end{tabbing}
} % Include the file specifying document layout
\usepackage{scrextend}

\begin{document}

%----------------------------------------------------------------------------------------
%	NAME AND CONTACT INFORMATION
%----------------------------------------------------------------------------------------

\title{Benjamin H Glick} % Print the main header

%------------------------------------------------

\parbox{0.5\textwidth}{ % First block
\begin{tabbing} % Enables tabbing
\hspace{3cm} \= \hspace{4cm} \= \kill % Spacing within the block
{\bf Address} \> 415 W Aldine Ave, Apt 10A\\ % Address line 1
\>Chicago, Il 60657  \\ % Address line 2
{\bf Nationality} \> American % Nationality
\end{tabbing}}
\hfill % Horizontal space between the two blocks
\parbox{0.5\textwidth}{ % Second block
\begin{tabbing} % Enables tabbing
\hspace{3cm} \= \hspace{4cm} \= \kill % Spacing within the block
%{\bf Home Phone} \> +1 (773) 975 8207 \\ % Home phone
{\bf Mobile Phone} \> +1 (312) 391 0727 \\ % Mobile phone
{\bf Email} \> \href{mailto:glick@lclark.edu}{glick@lclark.edu} \\ % Email address
{\bf Github} \> \href{https://github.com/benhg}{@benhg} \\ % Email address
{\bf Website} \> \href{https://glick.cloud}{https://glick.cloud} \\ % Email address
%{\bf ResearchGate} \> \href{https://www.researchgate.net/profile/Benjamin_Glick3}{/profile/Benjamin_Glick3} \\ % Email address
\end{tabbing}}

%----------------------------------------------------------------------------------------
%	Goals
%----------------------------------------------------------------------------------------

\section{Goals}

{To be a team cheerleader and tactician, to use my subject knowledge technical skills and interpersonal skills to take part in valuable projects.}

%---------------------------------------------------------------------------------------
%Strengths

\section{Strengths/Skills}

{Planning, problem solving, tactics, motivating or inspiring others, anchor of a team or group, ability to develop CS skills, speed at grasping new and complex concepts. Technical skills include python, c, c++,databases, javascript, html/css, java, functional programming including lisp and haskell, network programming, systems software, and \LaTeX/\TeX}

%----------------------------------------------------------------------------------------
%	EDUCATION SECTION
%----------------------------------------------------------------------------------------

\section{Education}

\tabbedblock{
\bf{2016-Present } \>  Student (BA, Physics \& Mathematics/Computer Science, Expected 2020) -\\\> \href{http://lclark.edu}{Lewis and Clark College} \\[5pt]
\>\+
\bf Relevant Coursework Includes \\
\> $\bullet$Computer Networks and Web Development \\
\> $\bullet$ Computer Architecture and Assembly Languages\\
\> $\bullet$ Computer Graphics \\
%\> $\bullet$ Physics Intro Sequence \\
\> $\bullet$ Linear Algebra \\
\> $\bullet$ Single-variable and Multi-variable Calculus \\
\> $\bullet$ Independent Study in High Performance Compute Job Optimization\\
\> $\bullet$ Discrete Mathematics\\
\> $\bullet$ Physics Sequence Including\\
\> \phantom{aaaaa}$\bullet$ Mechanics\\
\> \phantom{aaaaa}$\bullet$ Waves and Matter\\
\> \phantom{aaaaa}$\bullet$ Electricity and Magnetism\\
\> \phantom{aaaaa}$\bullet$ Thermodynamics\\
\> \phantom{aaaaa}$\bullet$ Experimental Methods in the Physical Sciences\\
\<
\bf{Mentors and Advisors}\\
\> $\bullet$Lab Mentor at University of Chicago/Argonne National Lab: Ian Foster \\
\> $\bullet$College Mentor and Advisor: Jens Mache \\
\<\\
\<
\bf{2012-2016} \> Student (High School Diploma, 2016) - \href{http://ucls.uchicago.edu}{The University of Chicago Laboratory High School}\\[-20pt]
}




%----------------------------------------------------------------------------------------
%	Experiences
%----------------------------------------------------------------------------------------

\section{Experience}

\job
{May 2019 -}{Aug. 2019}
{GE Transportation, A Wabtec Company}
{https://www.getransportation.com/home}
{Digital Technology Leadership Program Intern}
{}


\job
{Jan 2019 -}{May 2019}
{Lewis and Clark College}
{http://lclark.edu}
{Teaching Assistant}
{Teaching assistant for Lewis \& Clark's CS 495 Parallel and High Performance Computing My duties include teaching classes, holding office hours, providing advice to students, and assisting with design of coursework.}


\job
{May 2018 -}{Aug. 2018}
{General Electric Transportation}
{https://www.getransportation.com/home}
{Digital Technology Leadership Program Intern}
{Member of a production cybersecurity team. Designed prototype and proof of concept software and hardware systems. Designed and implemented software solutions for locomotive control computers and the GoLinc platform relating to data collection, encryption, management, and movement. Developed internal tools for security and cost audits of product teams.}

\job
{June 2017 -}{Present}
{Swift Project, Argonne National Laboratory / Computation Institute, University of Chicago}
{http://parsl-project.org}
{Research Software Engineer}
{The Swift research group creates and maintains high-performance computing tools for scientific and data-intensive computing. With Swift, I develop, maintain, and manage tools to make data-intensive and computationally demanding tasks easy to use, secure, and scalable in a variety of computing environments from multicore computers to some of the largest supercomputers in the world. I contribute to development of live projects with active scientific users as well as prototypes for future projects.}


\job
{Jan 2018 -}{Present}
{Watzek Library, Lewis and Clark College}
{https://library.lclark.edu/c.php?g=582711\&p=4023189}
{Digital Innovation Specialist}
{Watzek Digital Initiatives handles Lewis \& Clark's digital collections and infrastructure, as well as supporting research and academic computing on campus. As DI Specialist, I manage operation of LC's high-performance computing infrastructure, design solutions to help students, staff, and faculty solve digital problems, and assist in maintenance of the library's digital information resources.}


\job
{Jan 2018 -}{May 2018}
{Lewis and Clark College}
{http://lclark.edu}
{Teaching Assistant}
{Teaching assistant for Lewis \& Clark's CS 393 Computer Networks course. My duties include teaching classes, holding office hours, providing advice to students, and assisting with design of coursework.}

\job
{May 2017 -}{Aug. 2017}
{Knowledge Lab, University of Chicago / Harvard Business School}
{http://knowledgelab.org}
{Research Assistant}
{I designed, deployed, and analyzed a survey to scientists which was used to help understand how and why authors cite particular works and whether their citations can be used as a measure of performance in sicence. The survey used proprietary data on millions of scientists and their publications/citations.}
\job
{Jan. 2017 -}{May 2017}
{Lewis \& Clark College}
{http://lclark.edu}
{Stewart LLC Coordinator}
{I was responsible for community development in the Stewart residence hall Holistic Wellness Living Learning Community. I planned events, promoted discussion of community guidelines, and worked with the Campus Living staff to ensure that residents felt safe and happy in their residence hall.}
\job
{June 2016 -}{Aug. 2016}
{Globus/Computation Institute, University of Chicago}
{http://globus.org}
{Summer Intern}
{Worked on a team to develop a resource for the globus group at the University of Chicago. Devoloped a product which allows scientists to securely search, share, process and access confidential scientific data stored in the globus cloud.}

\job
{June 2015 -}{Aug. 2015}
{Computation Institute, University of Chicago}
{http://www.ci.uchicago.edu}
{RDCEP Summer Scholar}
{Developed and applied skills in Mathematics and Computer Science. Member of a small team of high school interns on projects using large scale data.   Developed ability to build my own solutions using mathematical models.  Navigate and organize a team with different skills and motivation levels}

%------------------------------------------------


%\job
%{June 2014 -}{July 2014}
%{Beijing No.4 High School}
%{http://www.bhsf.cn}
%{Student}
%{Studied Chinese Language and Culture at Beijing High School Four. Lived with a non English speaking host family, developed Chinese language and communication skills and learned about education and teenage life in China.  Immersed in a culture with few similarities to my own}

%\job
%{Aug. 2015 -}{June 2016}
%{University of Chicago Laboratory High School}
%{http://ucls.uchicago.edu}
%{Member, Varsity Soccer Team}
%{Joined team as a senior, learned how to be a valuable part of the team despite my soccer technical skills not being as good as others’. Developed tactics and helped to implement small tactical fixes.  Helped younger players adjust and gave  them someone to relate to.}

%\job
%{July 2009,} {2011, 2013}
%{Sanborn Western Camps}
%{http://www.sanbornwesterncamps.com}
%{Camper, Big Spring Ranch for Boys}
%{Learned to successfully live with little privacy around other people my age.  Learned camping and survival skills. Climbed mountains and learned leave no trace principles of camping.}


\section{Talks and Presentations}

\award
{May 2019}
{}
{Portland State University CS 406/506 Accelerated Computing}
{http://web.cecs.pdx.edu/~karavan/acc}
{Guest Lecture}
{Talk introducing use cases of GitHub Pages \\ \textbf{Glick B. H.} and Mache, J. "Introduction to OpenACC." Guest Lecture in Prof. Karen Karavanic's CS 406/506 Accelerated Computing with GPUs and Intel Phi, Spring 19. (May 2019)}

\award
{May 2019}
{}
{Lewis \& Clark College FTI}
{https://fti2019.sched.com/}
{Short Talk}
{Talk introducing use cases of GitHub Pages \\ \textbf{Glick, B.H.}"Jupyter Notebooks" Short Talk at Lewis \& Clark College FTI ( Faculty Technology Institute ) 2019. (May 2019)}

\award
{May 2019}
{}
{Lewis \& Clark College FTI}
{https://fti2019.sched.com/}
{Short Talk}
{Talk introducing use cases of Jupyter Notebooks \\ \textbf{Glick, B.H.}"Jupyter Notebooks" Short Talk at Lewis \& Clark College FTI ( Faculty Technology Institute ) 2019. (May 2019)}


\award
{Apr. 2019}
{ }
{Lewis \& Clark College ACM Student Chapter}
{https://acm.watzek.cloud}
{Workshop}
{Hands-on workshop introducing the fundamentals of robotics and embedded devices engineering through Poembot. \\ Bhatia, R.P. and \textbf{Glick, B.H.}  "Poembot: Behind the Scenes." Student Led Workshop hosted by Lewis \& Clark College ACM Student Chapter}

\award
{Apr. 2019}
{}
{ML4All 2019}
{https://ml4all.org}
{Lightning Talk}
{Talk on the uses of machine learning, as applied to the scheduling of high performance computing jobs. \\  \textbf{Glick, B.H.} "Everyone's Feeding Text. We feed Code (Machine Learning for Static Code Analysis)" (Lightning Talk). At ML4ALL 2019 (Apr. 2019)}

\award
{Apr. 2019}
{}
{Lewis \& Clark College Festival of Scholars and Artists, 2019}
{https://sc18.supercomputing.org}
{Talk}
{Talk on the uses of machine learning, as applied to the scheduling of high performance computing jobs. \\  \textbf{Glick, B.H.} "Applications of Machine Learning to High Performance Job Scheduling". In Program of the Festival of Scholars and Artists, 2019 (Apr. 2019).}

\award
{Jan. 2019}
{ }
{Lewis \& Clark College ACM Student Chapter}
{https://acm.watzek.cloud}
{Workshop}
{Hands-on bioinfomatics and biocomputing workshop on the basics of how to write code and draw conclusions from genomic and other biological data. \\ Somers, J. and  \textbf{Glick, B.H.} "Intro to Biocomputing." Student Led Workshop hosted by Lewis \& Clark College ACM Student Chapter}

\award
{Nov 2018}
{ }
{SC '18}
{https://sc18.supercomputing.org}
{Lightning Talk}
{Lightning talk for students about research in accessibility of high-performance computing systems to people of varying computational experience.}


\award
{May 2018}
{ }
{Lewis \& Clark College Faculty Technology Institute}
{https://www.lclark.edu/information_technology/client_services/faculty_technology_institute/}
{Oral Presentation}
{Talk introducing the applications and opportunities associated with High Performance Computing, focused on how LC faculty may be interested in HPC..\\  McWilliams, J. A., \textbf{Glick, B.H.}, Abbaspour, P. High Performance Computing at L\&C. In \textit{Program of Faculty Technology Institute, 2018} (May 2018). DOI: 10.13140/RG.2.2.35801.01125}


\award
{May 2018}
{ }
{Lewis \& Clark College Faculty Technology Institute}
{https://www.lclark.edu/information_technology/client_services/faculty_technology_institute/}
{Oral Presentation}
{Talk introducing the applications and opportunities associated with High Performance Computing, focused on how LC faculty may be interested in HPC..\\  McWilliams, J. A., \textbf{Glick, B.H.}, Abbaspour, P. High Performance Computing at L\&C. In \textit{Program of Faculty Technology Institute, 2018} (May 2018). DOI: 10.13140/RG.2.2.35801.01125}

\award
{Apr. 2018}
{ }
{Lewis \& Clark College Festival of Scholars and Artists}
{https://college.lclark.edu/festival-of-scholars/}
{Oral Presentation}
{Talk introducing the applications and opportunities associated with High Performance Computing, specifically relating to how and why  Lewis \& Clark students, staff and faculty might utilize HPC.\\ \textbf{Glick, B.H.} A Gentle Introduction to High Performance Computing. In \textit{Program of the Festival of Scholars and Artists, 2018} (Apr. 2018). DOI: 10.13140/RG.2.2.35801.01125}


\award
{Mar. 2018}
{ }
{Oregon Academy of Science}
{http://oregonacademyscience.org/meeting/}
{Oral Presentation}
{Work relating to Accessibile High-Performance Computing presented at Oregon Academy of Science annual meeting.\\ \textbf{Glick, B.H.} and Mache, J. Building an Accessible Web-Based Frontend for High-Performance Clusters. In \textit{Proceedings of the Oregon Academy of Science.} (Mar. 2018). DOI: 10.13140/RG.2.2.24328.11528}


\section{Publications}

\award{May 2019}
{}
{EduPar 19, at the IEEE International Parallel and Distributed Processing Symposium }
{https://tcpp.cs.gsu.edu/curriculum/?q=edupar19}
{Short Paper}
{Work relating to computational physics and research computing education published at the 33rd IPDPS. \\ 
	\textbf{Glick, B.H.} and Mache, J. "Finding the Electric Potential of a Square Wire" (Peachy Parallel Assignment). In \textit{Proceedings of The 9th NSF/TCPP Workshop on Parallel and Distributed Computing Education (EduPar-19)} (May 2019) }

\award
{Feb. 2019}
{}
{Oregon Academy of Science }
{http://oregonacademyscience.org/meeting/}
{Abstract and Oral Presentation (Forthcoming)}
{Abstract and talk describing a novel approach to HPC job scheduling, using machine learning to predict HPC job behavior in order to create a more intelligent execution schedule.\\ \textbf{Glick, B. H.} and Mache, J. "Using Machine Learning to Enable Job-Aware Scheduling". In \textit{Proceedings of the Oregon Academy of Science} (Feb. 2019)}


\award
{Nov. 2018}
{}
{SC '18/ EduHPC '18}
{https://sc18.supercomputing.org}
{Workshop Paper}
{Article describing high-performance computing workflow optimization platform, specifically designed to provide an HPC environment conducive to educational computing accepted to Workshop on Education and High Performance Computing 2018 (EduHPC 18), at SC '18. \\ \textbf{Ben Glick} and Jens Mache. 2018. \textit{Jupyter Notebooks and User-Friendly HPC Access} Workshop on High Performance Computing and Education, 2018 (EduHPC '18), at the International Conference for High Performance Computing, Networking, Storage and Analysis (SC '18)   (Nov. 2018). DOI 10.1109/EduHPC.2018.00005. \href{https://ieeexplore.ieee.org/document/8638386}{https://ieeexplore.ieee.org/document/8638386}}

\award
{Oct. 2018}
{}
{Consortium of the Computing Sciences in Colleges, Northwest Region.}
{https://ccsc-nw.org}
{Poster and Award}
{Poster describing computational platform for providing researchers and students with access to high-performance computing resources without requiring technical knowledge about the underlying HPC software and hardware presented at CCSC-NW meeting and won best student poster award for 2018 meeting.}


\award
{Oct. 2018}
{}
{Journal of Computing Sciences in Colleges}
{https://ccsc-nw.org}
{Journal Article}
{Article describing an open-source course curriculum and additional teaching materials published in J. Comput. Sci. Coll. \\ \textbf{Ben Glick} and Jens Mache. 2018. \textit{Using jupyter notebooks to learn high-performance computing.} J. Comput. Sci. Coll. 34, 1 (October 2018), 180-188. (Oct. 2018) \href{https://dl.acm.org/citation.cfm?id=3280518}{https://dl.acm.org/citation.cfm?id=3280518}}


\award
{Oct. 2018}
{}
{Consortium of the Computing Sciences in Colleges}
{https://ccsc-nw.org}
{Paper}
{Paper describing an open-source course curriculum and additional teaching materials accepted to the Northwest regional Conference of the Consortium of the Computing Sciences in Colleges. Paper presents an interactive course meant to be either taught or used on a self-guided basis. Paper will be published in Journal of the Computinc Sciences in Colleges.\\ \textbf{Glick, B.H.} and Mache, J. USING JUPYTER NOTEBOOKS TO LEARN HIGH-PERFORMANCE COMPUTING. In \textit{Proceedings of the Conference of the Northwest Regional Consortium of Computing Sciences in Colleges.} (Oct. 2018)}

\award
{Aug. 2018}
{ }
{The International Conference on Parallel Processing}
{https://www.researchgate.net/publication/324164543_An_Extensible_Ecosystem_of_Tools_Providing_User_Friendly_HPC_Access_and_Supporting_Jupyter_Notebooks}
{Paper}
{Poster and Paper describing an extensible ecosystem of accessibility tools for convenient HPC use without complex command line skills.\\ \textbf{Glick, B.H.} and Mache, J. An Extensible Ecosystem of Tools Providing User Friendly HPC Access and Supporting Jupyter Notebooks. In \textit{Proceedings of The International Conference on Parallel Processing.} (Aug. 2018). http://oaciss.uoregon.edu/icpp18/views/includes/files/pos107s1-file1.pdf.}

\award
{Mar. 2018}
{ }
{Oregon Academy of Science}
{http://oregonacademyscience.org/meeting/}
{Abstract for Oral Presentation}
{Work relating to Accessibile High-Performance Computing published at Oregon Academy of Science.\\ \textbf{Glick, B.H.} and Mache, J. Building an Accessible Web-Based Frontend for High-Performance Clusters. In \textit{Proceedings of the Oregon Academy of Science.} (Mar. 2018). DOI: 10.13140/RG.2.2.24328.11528}

\award
{Nov. 2017}
{ }
{IEEE and ACM SIGHPC}
{https://sc17.supercomputing.org}
{Poster}
{Work relating to cloud computing infrastructure published at SC 2017. \\ \textbf{Glick, B.H.}, Babuji, Y.N., and Chard, K. 2017. Scalable Parallel Scripting in the Cloud. In \textit{Proceedings of the International Conference for High Performance Computing, Networking, Storage and Analysis (SC '17).} (Nov. 2017). 2 Pages. DOI: 10.13140/RG.2.2.20048.81922}

\section{Awards, Group Memberships, and Recognition}

\award
{Jan. 2019}
{Present}
{Lewis \& Clark College}
{https://lclark.edu}
{Membership}
{Joined a curricular development committee responsible for designing, implementing, and marketing a new interdisciplinary data sciences program including a new data science major, a new data science minor, and new data and computational sciences center.}

\award
{Sep. 2018}
{Present}
{Lewis \& Clark College}
{https://lclark.edu}
{Membership}
{Joined a curricular development committee responsible for redesigning and reimplementing natural sciences requirements at Lewis \& Clark College.}

\award
{Sep. 2017-}
{Present}
{Python Software Foundation}
{https://python.org}
{Membership}
{Became a contributing member (voting membership) of the Python Software Foundation by invitation.}

\award
{Sep. 2017}
{ }
{Lewis \& Clark College}
{ }
{Special Selection}
{Selected to be a member of the Lewis \& Clark College Council on Advanced Research in Data Science}

\award
{Jun. 2017-}
{Present}
{The Association of Computing Machinery}
{https://acm.org}
{Membership}
{Became a member of the Association for Computing Machinery and ACM SIGHPC Special Interest Group in High-Performance Computing.}

\award
{May. 2017-}
{Present}
{Institute of Electrical and Electronics Engineers}
{https://ieee.org}
{Membership}
{Became a member of the Institute of Electrical and Electronics Engineers (IEEE)}

\award
{May 2017}
{ }
{Lewis \& Clark College}
{}
{Award}
{Dean's list, 2017 Spring.}

\award
{April 2017}
{ }
{National Cyber League}
{}
{Competition}
{Placed 60th (out of 2000) in the 2017 Spring National Cyber League cybersecurity competition.}

\award
{April 2017}
{ }
{National Cyber League Team Competition}
{}
{Competition}
{Placed 38th in the 2017 Spring National Cyber League team cybersecurity competition.}

\award
{Feb. 2017}
{ }
{Pacific Rim Regional Cyber Defence Competition}
{}
{Competition}
{Placed 3rd in the 2017 Pacific Rim Regional Cyber Defence Competition.}
%----------------------------------------------------------------------------------------
%	INTERESTS SECTION
%----------------------------------------------------------------------------------------

\section{Activities}
\activity
{Jan. 2018 -}{Present}
{Lewis \& Clark College Student ACM Chapter}
{http://lclark.edu}
{Founder and Chair}
{I am one of the founding board members of the Lewis \& Clark College ACM Student Chapter. The ACM Chapter oversees all extracurricular computer science related activity, including organizing teams for competitions, hosting speakers and career events, and a colloquium series on topics in the computing sciences.}


\activity
{Jan. 2018 -}{Present}
{Lewis \& Clark College Fire Arts Club}
{http://lclark.edu}
{Vice President and Financial Officer}
{I am an officer of the Fire Arts Club at Lewis \& Clark. I am responsible for planning and safely executing fire arts performances, ensuring that our club is a safe and welcoming environment, and ensuring that the budget remains balanced.}


\activity
{Sep. 2004 -}{Present}
{Brett Wolf Judo/Menomonee Judo Club}
{http://brettwolfjudo.com}
{Senpai/Judoka}
{Competed at National Level, Taught beginner students, disabled students, and military veterans. Learned leadership skills through mentoring and coaching younger teammates.  Learned to teach in an adaptive way from working around people’s various abilities. Developed ability to approach events in my life with a belief system that helps me have empathy and integrity.}

\activity
{Sep. 2015 -}{Jun. 2016}
{University of Chicago Laboratory High School}
{http://ucls.uchicago.edu}
{Board Member, Computer Science Club (Code@Lab)}
{Founding board memeber of the U-High computer science club (code@lab). Responsible for planning club activities and meetings, as well as engaging and recruiting members.}

\activity
{Aug. 2015 -}{June 2016}
{University of Chicago Laboratory High School}
{http://ucls.uchicago.edu}
{Member, Varsity Soccer Team}
{Joined team as a senior, learned how to be a valuable part of the team despite my soccer technical skills not being as good as others’. Developed tactics and helped to implement small tactical fixes. helped younger players adjust and gave  them someone to relate to.}


\end{document}